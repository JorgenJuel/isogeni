
Husk fra definisjonen av morfier at en isogeni derfor består av funksjoner i $\overline{K}(E)$, altså $\phi = [f_1, f_2, f_3]$ med $f_i \in \overline{K}(E)$. På samme måte som tidligere er vi her også interessert i "K-rasjonale" funksjoner, eller funksjoner definert over $K(E)$ i steden. Siden $\phi^\sigma = [f_1^\sigma, f_2^\sigma, f_3^\sigma]$ har vi at en isogeni, $\phi$ er definert over $K$ hvis og bare hvis $\phi^\sigma = \phi$ for alle $\sigma \in \overline{K}/K$.

\textbf{TODO: Vis dette mot $GL_2(\ZZ)$ eller hva enn dette var} 

En interessant egenskap ved isognier er at de kan defineres ut i fra en undergruppe av $E(K)$. 


Merk at $E'$ ikke er restklassen til $E/\Phi$, men at $E/\Phi \subset E'$ dersom man velger \textit{riktige} representanter for punktene. $E(K)$ og $E'(K)$ kommer til å ha like mange elementer dersom $\phi$ er $K-rasjonal$. Dette følger av at selve avbildingen ikke er begrenset til de $K-$rasjonale punktene, selv om vi ser på den. Altså vil selve avbildingen gå fra kurven $E$ til kurven $E'$.

\begin{eksempel}
La $y^2 - x^3 - 5$ være en elliptisk kurve over $\FF_7$. Der $E(\FF_7) = \{(3,2), (3,5), (5,2), (5,5), (6,2), (6,5), \OO = (0,1) \}$ - altså $\#E(\FF_7) = 7$.
\end{eksempel}
         	 	
\begin{definisjon}
Graden til en isogeni, $deg(\phi)$ med $\phi: E \rightarrow E'$, er graden av den endelige utvidelsen $\overline{K}(E)/\phi^*\overline{K}(E')$.
\end{definisjon}


Men for vår del er det ikke bare nok å ha en isogeni over en vilkårlig kropp. På samme måte som vi ønsker oss de $\FF_q$-rasjonelle punktene til en elliptisk kurve vil vi også se på isogeniene definert over $\FF_q$.

Øvelse 3.13: